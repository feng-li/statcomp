\documentclass{article}
\usepackage{bm}
\begin{document}
\title{Statistical Computing: Simulating Random Variables}
\date{}
\maketitle
\section*{Direct Method}
\begin{enumerate}
\item Simulate 1000 variables using the direct method for
\begin{enumerate}
\item The Weibull Distribution ($a=0.5,\,b=2$) CDF: $F(x)=1-e^{-(x/b)^a}$
\item The Logistic Distribution CDF: $F(x)=(1+e^{-x})^{-1}$
\end{enumerate}
You must work out and code up the inverse cdf function yourself
\item Do the same using the R functions {\em rweibull} and {\em rlogis}.  Compare the results to your answer in Question 1 by looking at summary statistics using the {\em summary} function in R.
\end{enumerate}
\section*{Indirect Method}
\begin{enumerate}
\item Use the Accept/Reject algorithm to simulate from the standard lognormal distribution which has density $f(x)=(2\pi x^2)^{-1/2}e^{-log(x)^2/2}$.  Issues to think about:  What can be used as a proposal?  How to select M.
\item Use the Accept/Reject algorithm to simulate from a $N(0,1)$.  You can use the function {\em dnorm}. Try the following proposals.
\begin{enumerate}
\item Logistic Distribution (use {\em rlogis} and {\em dlogis} functions).
\item Cauchy distribution (use {\em rcauchy} and {\em dcauchy} functions).
\item Student t distribution 5 df (use {\em rt} and {\em dt} functions).
\item Student t distribution 20 df (use {\em rt} and {\em dt} functions).
\item A N(1,2) distribution (use {\em rnorm} and {\em dnorm} functions)
\end{enumerate}
\item For each proposal record the percentage of iterates that are accepted.  Which is the best proposal in Question 2? Why?
\end{enumerate}
\end{document}
